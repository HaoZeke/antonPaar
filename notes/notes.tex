% Created 2019-09-23 Mon 15:32
% Intended LaTeX compiler: xelatex
\documentclass[12pt,a4paper,oneside,headinclude]{scrartcl}
\usepackage{graphicx}
\usepackage{grffile}
\usepackage{longtable}
\usepackage{wrapfig}
\usepackage{rotating}
\usepackage[normalem]{ulem}
\usepackage{amsmath}
\usepackage{textcomp}
\usepackage{amssymb}
\usepackage{capt-of}
\usepackage{hyperref}
\usepackage{minted}
\PassOptionsToPackage{unicode=true}{hyperref}
\PassOptionsToPackage{hyphens}{url}
\PassOptionsToPackage{dvipsnames,svgnames*,x11names*,table}{xcolor}
\usepackage{lmodern}
\usepackage{amssymb,amsmath}
\usepackage{physics}
\usepackage{ifxetex,ifluatex}
\usepackage{fixltx2e} % provides \textsubscript
\ifnum 0\ifxetex 1\fi\ifluatex 1\fi=0 % if pdftex
\usepackage[T1]{fontenc}
\usepackage[utf8]{inputenc}
\usepackage{textcomp} % provides euro and other symbols
\else % if luatex or xelatex
\usepackage{unicode-math}
\defaultfontfeatures{Ligatures=TeX,Scale=MatchLowercase}
\fi
% use upquote if available, for straight quotes in verbatim environments
\IfFileExists{upquote.sty}{\usepackage{upquote}}{}
% use microtype if available
\IfFileExists{microtype.sty}{%
\usepackage[]{microtype}
\UseMicrotypeSet[protrusion]{basicmath} % disable protrusion for tt fonts
}{}
\IfFileExists{parskip.sty}{%
\usepackage{parskip}
}{% else
\setlength{\parindent}{0pt}
\setlength{\parskip}{6pt plus 2pt minus 1pt}
}
\usepackage{hyperref}
\hypersetup{
pdftitle={Surface Area and Porous Material Characterization},
pdfauthor={Rohit Goswami},
pdfborder={0 0 0},
breaklinks=true}
\urlstyle{same}  % don't use monospace font for urls
\usepackage{longtable,booktabs}
% Fix footnotes in tables (requires footnote package)
\IfFileExists{footnote.sty}{\usepackage{footnote}\makesavenoteenv{longtable}}{}
\usepackage{graphicx,grffile}
\makeatletter
\def\maxwidth{\ifdim\Gin@nat@width>\linewidth\linewidth\else\Gin@nat@width\fi}
\def\maxheight{\ifdim\Gin@nat@height>\textheight\textheight\else\Gin@nat@height\fi}
\makeatother
% Scale images if necessary, so that they will not overflow the page
% margins by default, and it is still possible to overwrite the defaults
% using explicit options in \includegraphics[width, height, ...]{}
\setkeys{Gin}{width=\maxwidth,height=\maxheight,keepaspectratio}
\setlength{\emergencystretch}{3em}  % prevent overfull lines
\providecommand{\tightlist}{%
\setlength{\itemsep}{0pt}\setlength{\parskip}{0pt}}
\setcounter{secnumdepth}{0}
% Redefines (sub)paragraphs to behave more like sections
\ifx\paragraph\undefined\else
\let\oldparagraph\paragraph
\renewcommand{\paragraph}[1]{\oldparagraph{#1}\mbox{}}
\fi
\ifx\subparagraph\undefined\else
\let\oldsubparagraph\subparagraph
\renewcommand{\subparagraph}[1]{\oldsubparagraph{#1}\mbox{}}
\fi
% Make use of float-package and set default placement for figures to H
\usepackage{float}
\floatplacement{figure}{H}
\numberwithin{figure}{section}
\numberwithin{equation}{section}
\numberwithin{table}{section}
\makeatletter
\@ifpackageloaded{subfig}{}{\usepackage{subfig}}
\@ifpackageloaded{caption}{}{\usepackage{caption}}
\captionsetup[subfloat]{margin=0.5em}
\AtBeginDocument{%
\renewcommand*\figurename{Figure}
\renewcommand*\tablename{Table}
}
\AtBeginDocument{%
\renewcommand*\listfigurename{List of Figures}
\renewcommand*\listtablename{List of Tables}
}
\@ifpackageloaded{float}{}{\usepackage{float}}
\floatstyle{ruled}
\@ifundefined{c@chapter}{\newfloat{codelisting}{h}{lop}}{\newfloat{codelisting}{h}{lop}[chapter]}
\floatname{codelisting}{Listing}
\makeatother
\usepackage[dvipsnames,svgnames*,x11names*,table]{xcolor}
\definecolor{listing-background}{HTML}{F7F7F7}
\definecolor{listing-rule}{HTML}{B3B2B3}
\definecolor{listing-numbers}{HTML}{B3B2B3}
\definecolor{listing-text-color}{HTML}{000000}
\definecolor{listing-keyword}{HTML}{435489}
\definecolor{listing-identifier}{HTML}{435489}
\definecolor{listing-string}{HTML}{00999A}
\definecolor{listing-comment}{HTML}{8E8E8E}
\definecolor{listing-javadoc-comment}{HTML}{006CA9}
\usepackage{pagecolor}
\usepackage{afterpage}
\setcounter{tocdepth}{3}
\usepackage{setspace}
\setstretch{1.2}
\usepackage{csquotes}
\usepackage[font={small,it}]{caption}
\newcommand{\imglabel}[1]{\textbf{\textit{(#1)}}}
\definecolor{blockquote-border}{RGB}{221,221,221}
\definecolor{blockquote-text}{RGB}{119,119,119}
\usepackage{mdframed}
\newmdenv[rightline=false,bottomline=false,topline=false,linewidth=3pt,linecolor=blockquote-border,skipabove=\parskip]{customblockquote}
\renewenvironment{quote}{\begin{customblockquote}\list{}{\rightmargin=0em\leftmargin=0em}%
\item\relax\color{blockquote-text}\ignorespaces}{\unskip\unskip\endlist\end{customblockquote}}
\definecolor{heading-color}{RGB}{40,40,40}
\addtokomafont{section}{\color{heading-color}}
\usepackage{titling}
\renewcommand{\arraystretch}{1.3} % table spacing
\definecolor{table-row-color}{HTML}{F5F5F5}
\rowcolors{3}{}{table-row-color!100}
% Reset rownum counter so that each table starts with the same row color
\let\oldlongtable\longtable
\let\endoldlongtable\endlongtable
\renewenvironment{longtable}{\oldlongtable} {
\endoldlongtable
\global\rownum=0\relax}
\setlength{\parindent}{0pt}
\setlength{\parskip}{6pt plus 2pt minus 1pt}
\setlength{\emergencystretch}{3em}  % prevent overfull lines
\usepackage{fancyhdr}
\pagestyle{fancy}
\fancyhead{}
\fancyfoot{}
\lhead{Surface Area and Porous Material Characterization by Dr. Thomas}
\chead{}
\rhead{\today}
\lfoot{Compiled by Rohit Goswami \textsc{\scriptsize\ MInstP AMIChemE AMIE}}
\cfoot{}
\rfoot{\thepage}
\renewcommand{\headrulewidth}{0.4pt}
\renewcommand{\footrulewidth}{0.4pt}
% When using the classes report, scrreprt, book,
% scrbook or memoir, uncomment the following line.
%\addtokomafont{chapter}{\color{heading-color}}
\usepackage[default]{sourcesanspro}
\usepackage{sourcecodepro}
\usepackage[margin=2.5cm,includehead=true,includefoot=true,centering]{geometry}
\author{Rohit Goswami\textsc{\scriptsize\ MInstP AMIChemE AMIE}}
\date{\today}
\title{Surface Area and Porous Material Characterization\\\medskip
\large Notes from Martin Thomas' Short Course}
\hypersetup{
 pdfauthor={Rohit Goswami\textsc{\scriptsize\ MInstP AMIChemE AMIE}},
 pdftitle={Surface Area and Porous Material Characterization},
 pdfkeywords={},
 pdfsubject={},
 pdfcreator={Emacs 26.3 (Org mode 9.1.9)}, 
 pdflang={English}}
\begin{document}

\begin{titlepage}
\newgeometry{left=6cm}
\definecolor{titlepage-color}{HTML}{06386e}
\newpagecolor{titlepage-color}\afterpage{\restorepagecolor}
\newcommand{\colorRule}[3][black]{\textcolor[HTML]{#1}{\rule{#2}{#3}}}
\begin{flushleft}
\noindent
\\[-1em]
\color[HTML]{ffffff}
\makebox[0pt][l]{\colorRule[ffffff]{1.3\textwidth}{1pt}}
\par
\noindent

{ \setstretch{1.4}
\vfill
\noindent {\huge \textbf{\textsf{Surface Area and Porous Material Characterization}}}
\vskip 1em
{\Large \textsf{Notes from Martin Thomas' Short Course}}
\vskip 2em
\noindent
{\Large \textsf{\MakeUppercase{Rohit Goswami,\textsc{\scriptsize\ MInstP AMIChemE AMIE}}}
\vfill
}

\textsf{\today}}
\end{flushleft}
\end{titlepage}
\restoregeometry

\tableofcontents
\newpage

\section{Lec I: Physisorption and Basic Measurements}
\label{sec:org0e72d53}
Dr. Martin A. Thomas is the speaker for the day.
\begin{itemize}
\item Surface Area
\item Catalysis
\end{itemize}
\subsection{Essential Techniques}
\label{sec:org43c0fba}
\begin{itemize}
\item Surface area
\item Pore size distribution
\item Catalyst active area/dispersion
\item TPR and other flow materials
\end{itemize}
\subsection{Surface Area}
\label{sec:org670977b}
For the purposes of this scale, the geometric extent of the interface between
two materials in any phase would be the surface area. Surface area measurement
is useful for a variety of situations:
\begin{itemize}
\item Reveals an excess of fines that PSA cannot
\item Can distinguish between different morphologies of similar size
\item Can follow the generation of particle-particle bonds
Typically van-der-waals but under certain temperature and pressure can cause sintering which would be "true" bonds in
\end{itemize}
terms of ionic and covalent bonds.
\begin{itemize}
\item Temporal morphological stability
\item Confirms brittle fracture under compression
\item Tighter control on specifications of raw materials to avoid surprises during
later processing
\item Can explains unxepected dissolution behavior
\end{itemize}
\subsubsection{Discussion}
\label{sec:orgbb06043}
The PSA (particle size analysis) is not always of much use. Even the 90\%
median measures may miss out on certain complexities which broaden or alter the
spread without changing the median measures. The surface area considerations are
significant then. Of course the surface area is most important when one
considers porous materials. The surface as defined here extends through the bulk
of the material due to the porosity.
"Stickiness" is simply when the gravitational pull on the particle cannot
overcome the molecule's attraction to the surface. The surface free energy is
the amount avaliable for work. At the surface when the surface free energy
increases, there is an increase in the dissolution rate. When the surface area
decreases, the commiserate reduction in the surface free energy reduces
dissolution while increasing strength. Especially when there is a liquid, the
surface area is a very good measure of the liquid lost to the porosity.

\subsection{Dissolution rate}
\label{sec:orgf0c1078}
The Noyes-whitney or the nerst-brunner equation is used for the dissolution rate:

$$\frac{dW}{dt}=\frac{DA(C_S-C)}{L}$$
\subsection{Pore Size}
\label{sec:org18127c3}
The pore size distribution is inversely proportion to the fluid adsorption.
\subsubsection{Hierarchical Pore Stuctures}
\label{sec:orgc1ca14b}
Inspired by the lungs, the idea is to have progressively larger surface area
with increasing depth.
\subsubsection{Catalyst Turnover}
\label{sec:org3389dbf}
A general trend in the turnover, that is, the number of reactions which may
occur at a reactive site is seen to be strongly correlated to the pore-size.
\subsection{Gas Adsorption}
\label{sec:orgedccf22}
This is the only technique which directly mesasures and quantifies the surface
area accurately. Hg intrusion is important too since it covers a wider range but
then it isn't a really direct measure.
The only biase is the sieving effect.

IUPAC definitons:
\begin{itemize}
\item Micropore is less than 2nm
\item Mesopore is between two to fifty nm
\item Macropores are larger than 50nm
\end{itemize}
\subsubsection{Calculation mMethods}
\label{sec:org5499efe}
A small pore version is a stronger absorber. The pore causes the adsorption
force to "focus" in a sense. LJ potential energy analysis shows that the
entering molecules "see" more than one surface during adsorption. The isotherm
is used to directly measure the gas interacting with a surface and within the pore.

Gas adsorption goes upto 500nm. The BJH is a classical method based on the
Kelvin equation. There are also DFT kernel methods.
Smaller pore has lower pore filling pressure. The pore filling pressure is used
to calculate the pore size and the extent of the surface. This also means we can
get the pore volume. The pore size distribution is a volume distribution, not a
spatial distribution.
\begin{enumerate}
\item Basics
\label{sec:org643c0a2}
We need to remember to have the saturation pressure in mind for physisorption.
\end{enumerate}
\subsubsection{Total Physisorption}
\label{sec:org41e7d2b}
This analyzes extensive ranges of physical properties. The BET surface area is
the Brunnet-Emmet-Teller method. Typically the BET surface area is one of the
total area. The T plot method is used in the multi-layer portion and is used to
get the micropore distribution. The mesopore size distribution shows a hysteris
curve from which information about the pore structure can be \textbf{qualitatively} obtained

\subsection{Flow Analysis}
\label{sec:org6a7eba9}
Cooling is important to get the information required. The desorption peak is
used to measure BET areas due to quick heating of the sample and get a sharper peak.
\subsubsection{Vacuum Volumetric Principle}
\label{sec:org79bb070}
This is a manometric principle. The point is still to obtain the isoterm. So the
construction involves a sample cell, cooled with liquid nitrogen. There is a
evacuation system to ensure that the full isotherm can be obtained. The
temperature (measured), volume (calibrated), and the pressure (controlled) in the manifold, the ideal
gas law can be used to figure out the amount absorbed.

Typically for surface area and pore size analysis:
\begin{itemize}
\item Nitrogen at 77K. For micropore size distribution for microporous carbons will
take upto 2 hours. For zeolites, argon at 87K is about twice as fast.
\item Argon at 87K. Is recommended by the IUPAC. This is very good for micropores
because it is monoatomic, no orientation issues, is non-polar and inert
(surface changes have no effect). Ar (87K) kinetics are faster than nitrogen
at 77K (shorter equilibration times).
\item CO\(_2\) at 273K. The polarization makes it unsuitable for zeolites. It is
typically used for narrow pores in microporous carbons. Can get a micropore
carbon in a few hours (4 to 6). A lot of this is because of the increased temperature.
\item Krypton 77K or 87K for very low temperatatures since it is really expensive.
Used when Nitrogen cannot be used.
\end{itemize}
\subsection{Practical Rules of Thumb}
\label{sec:org16a326b}
\subsubsection{Sample Cell Selection}
\label{sec:org534c83d}
The cell which minimizes the void volume is the right one. Overfilling risks
elutriation. Larger void volumes will cause a loss in precision.
 The smallest bulb which accomodates the optimal TSA but no more than two thirds full.
\subsubsection{Minimize negative impact of cold volume}
\label{sec:org6c74ed3}
Coolant level controlled at the top creates a large cold zone. Coolant level
controlled at the bottom causes a smaller cold zone.
\subsubsection{Sample Preparation}
\label{sec:org3c6008b}
Degassing is normally done by heating under flow, or by using a vacuum. The highest temperature which
will not ruin the material is what should be used. Ensure that the substance
does not degrade. Flow methods will not be very good in some cases, namely
because there is a water molecule in the pore which has to come out slowly.
\subsubsection{Degassing Time}
\label{sec:orgf5ed340}
The complete time is by experimentation. Typically its sixteen hours, which is
considered to be overnight. Samples that require low temperatures generally
require the longest outgas times.
\subsubsection{P/Po Range Selection}
\label{sec:org8a7bddd}
The multi-point BET is lower than the single point BET.
\subsubsection{Micropore challenges}
\label{sec:orgd1c5b42}
Typically there are many extended range micropore stations each with a dedicated
manifold and a complete set of transducers.
\subsubsection{Data Reduction}
\label{sec:org0ccee62}
Micropores are upto the leveling out of the isotherms. Multilayer use the Tplot.
The mesopore region is beyond that. Right at the top end is where the gas
absorbs into the interparticle space. This is essentially a feature of the
undefined interparticle sizes.

IUPAC pores:
\begin{itemize}
\item Pores smaller than 2nm
\item Supermicropores 0.7 to 2nm
\item Ultramicropores are less than 0.7nm
\end{itemize}

The density of the absorbed gas is estimated to figure out geometric aspects of
the pore being studied. Type four regions are essentially the macropore range.
For micropores you need the data from ten to the power minus ten to see the
micropore filling.

For different P/Po the same micropores may be observed. Nitrogen's strong
quadrupole shifts the pore filling to much lower P/Po. DFT can correct somewhat,
but argon is much more reliable.
Micropores usually have a turbo pump and a diaphragm pump.

The take away is that pore filling pressure is directly proportional to the pore
size diameter.

\begin{itemize}
\item Nitrogen is great for good for SA, Mesopore and total pore volume but not for
\end{itemize}
micropore sizes.
\begin{itemize}
\item Argon is recommended in all cases greater than 0.5nm because the gases listed
are least affected by surface chemistry.
\item Carbon dioxide cannot do anything other
\end{itemize}

\subsection{DFT}
\label{sec:orgfdbfc54}
Lowell, sheilds thomas and Thommes. Characterization of porous solids and
powders.
Neimark, Ravikovitch, Thammes, Carbon (2009)
Thommes Cychosz, Neimark

NLDFT skernel of metastable isotherms take into accound delays which may occur.

M. Thommes. In Nanoporous Materials Science and engineering

When there is a network defect, then there are ink-bottle pores, M. Thommes, B.
Smarsly, M. Grenewold, Ravikovitch, Neimark, Langmuir 2001.

\begin{quote}
Rutgers University under Alex Neimark carries out much of the academic DFT calculations
\end{quote}

\begin{quote}
For these systems, though earlier studies have used MC, DFT is actually faster.
\end{quote}

\begin{quote}
The temperature effects (local heating due to the exothermic nature of
adsorption) does cause a delay for accuracy, that is, the system is to be
equilibrated and cannot be hastened.
\end{quote}


\section{Lec 2: Chemisorption}
\label{sec:org47735fb}
This is also known asn reactive gas adsorption. This is typically used to
quantify the number of reactive sites. Silver catalysts are low dispersion.

\begin{table}[htbp]
\caption{Chemisorption}
\centering
\begin{tabular}{ll}
Feature & Analysis\\
Active Metal Area & TPR\\
Dispersion & TPO\\
Crysallite size & TPD\\
Acid site concentration & Activation Energy\\
Heat of adsorption & \\
\hline
\end{tabular}
\end{table}

The crystallite size is typically not the true size, except in some special
systems. Protonated zeolites have acidic sites, and that is the visual metric
used expressed as the amount of gas. There is no real conversion as there is no
real surface area calculation.

A wetting syste is such that the metal to the support affinity is higher than
that of the metal-metal affinity. There may be steric hinderance, but
essentially the wetting systems tend towards to spreading out over the surface.
When there is less dispersion, the metal will tend to coagulate in lumps, which
will cause some of the atoms to be not be accessible for reactions.

Physorption areas and chemisorption areas are different. The cross sectional
area is used for the physisorption, that is some 16 angstrom. The metal area
does not use the area of the adsorbed gas. It uses the size of the absorbing
site. The internuclear distance for the metal can also be used. The projected
area estimates can vary.

Forming a chemical bond does not require cryogenic conditions. Hydrogen is very
popular since it chemisorbs to a lot of metals. Since it forms a hydride
sometimes, there is some "solubility" in the metal, so the adsorbed gas may not
actually be localized on the surface. Typically where hydride formation is an
issue, CO is used. Twice as many CO moleculas are adsorbed compared to H\(_{\text{2}}\) for
Platinum. Multiple isotherms can be used to get the heat of adsorption and
correlate them to the acid site strength.

\begin{table}[htbp]
\caption{Metals and Reactive Gases for Chemisorption}
\centering
\begin{tabular}{lll}
Metal & Reactive gas & Comment\\
Pt & hydrogen, CO & Twice as many Co as H\(_{\text{2}}\)\\
Pd & CO & Pd forms bulk hydrides, i.e. not limited to the surface\\
Ni & Hydrogen & Ni forms carbonyls, not limited\\
Fe & Hydrogen & Fe forms carbonyles, not limit to the surface\\
Ag & Oxygen & At elevated temperatures\\
\hline
\end{tabular}
\end{table}

Ni may also be volatile enough to form a Ni mirror somewhere cooler.

\subsection{Stochiometry}
\label{sec:org7a84db3}
The number of metal atoms per gas molecule. The instrument can measure how much
gas is adsorbed. Normally it is assumed that hydrogen dissociates on the
surface, so one hydrogen molecule will dissociate over two metal atoms. In some
cases in the literature they use the number of atoms reating per molecule, so
that is half. When we consider CO, it may be one or bridged, and therefore two.
At low pressures of Co, the bridged system is more likely, so as the pressure
increases, it is reasonable to assume a steric geometry of one.
\subsection{Setup and Quantification}
\label{sec:org3afa21a}
The volumetric version of the equipment is usually the same as that used for
physisorption. A flow of reactive gas is used to get rid of water as well.

We try to stay close to the critical temperature for chemisorption studies.
In theory, after a layer is chemisorbed there is only pressure created over the
surface with the addition of gas after saturation. The first isotherm is
generated, then the system is evacuated, and another isotherm is plotted. Now
since we have not changed the temperature, to remove the extra effects which may
be in the system, the second isotherm is the weak, reverible isotherm. Thus we
can not calculate the strong or irreversibly adsorbed isotherm. To get the
amount we can do a simple least squares extrapolation to get the volume
adsorbed. The strongly adsorbed isotherm is considered to be pressure
independent. A cheisorption isotherm will typically have a much smaller number
of points, since it's essentially a straight line generation. Ten is a
reasonable default. It's much faster than physisorption of course. The exotherm
in chemisorption is unlikely to cause any changes while for physisorption a the
exothermic energy might cause the neighboring molecules to pop off. Close to the
desorption temperature there might be some curvature for chemisorption as well.

\subsubsection{Flow Methods}
\label{sec:orgdb6e22e}
Pulse titration is also used. Typically, the gas is pulsed (fixed volumes are
injected, typically 100 micro liters), in a carrier which is typically N\(_{\text{2}}\) or
He. Helium is common for carbon systems. Downstream, a TCD is used to analyze
the rest of the stuff. The process is isothermal but an isotherm is not
generated. Reversible adsorption esorbs after the puslse has passed. The weakly
adsorbed gases are swept away, and so not reversible equilibruim with the rest
of the surface. The TCD reacts to what was not taken up by the sample. Peak
integration is done to figure out how much was adsorbed. The difference between
the peak areas is used to figure out how much has been adsorbe. Sometimes the
peaks overlap when there isn't enough of a delay between the pulses. Weaker
pulses have more tailing. Upto six partial peaks are usually relevant.
For some systems like Pd/Ni where H and CO might not be a good fit to either Pd
or Ni, typically both are used. Flow techniques are also better for such systems
too. In bimetallics it is very difficult to create situations where one is
adsorbing and the other is not.
\subsection{Sample Preparation}
\label{sec:org749e52e}
The system has to be activated. Not just dried. Most samples are oxidized metals
so they need to be reduced with a suitable gas to the active state at elevated
temperatures, and then it needs to be purged and cooled. This is usually in a
u-shaped flow pattern cell. Unlike physisorption, preparation is done in-situ at
the analysis port under flowing gas conditions. The setup is easier for
chemisorption than physisorption in terms of protocols.

\subsubsection{Flow Methods}
\label{sec:org8de5b1a}
Typically 300K might be a good start, but not always. Nickel luminate needs
around 600K. Most plants which use steam reforming don't have temperatures that
high, so any nickel-luminate created is lost. So with a TPR you can figure out
what you can't from just a total nickel analysis.
\begin{itemize}
\item Oxide reducibility by TPR
\item Carbon oxidaion
\item Acid site concentration by TPD
\item Activation Energy
\end{itemize}

You can determine the nature of an acide ste by swamping the system in ammonia,
then heat it until it gets off.

\subsubsection{TPR (Temperature Promgrammed Reduction)}
\label{sec:org65d1b58}
Water is the main product of the reaction and can be removed from the gas flow
using a cold trap prior to the detector for a clean TCD signal. The exact shape
of the distribution depends on the amount of sample as well.
\begin{itemize}
\item Cerium IV oxide
\item Nickel oxide
\item Iron oxide
\item Cobalt oxide
\end{itemize}
Aluminum oxide won't reduce even at maximum instrument temperature.
\subsubsection{TPO (Temperature Programmed Oxidation)}
\label{sec:org16e0880}
Similar to TPR in that a gas mixture dilute in the reacting gas is passed over
the sample as the temperature is increased in a liner fasion. The main products
are CO and CO\(_{\text{2}}\). These are not trapped and allowed to go through.
\begin{itemize}
\item Carbon Nanotubes
\item Cerium III oxide
\item Coke on catalysts
\item Pre-reduced metals
\end{itemize}
Spikes in these curves may be due to the catalyst being left in the product.
\subsubsection{TPD (Temperature Programmed Desorption)}
\label{sec:org712d81d}
Can also be decomposition. The only way to be really sure of all the products is
a down-stream mass spectrometer but this is typically overkill.
\begin{itemize}
\item Acid sites (NH\(_{\text{3}}\))
\item Basic sites (with CO\(_{\text{2}}\))
\item Methanol from reactive oxides
\item Hydrogen form hydrides
\end{itemize}
It is assumed that they are Bronsted sites, but is not typically proven with
pyridine methods.
\section{Lec 3: Special Topics, Water}
\label{sec:orge1efda9}
Water absorption is interesting due to the difficulty in classication in terms
of the ambigous nature of the bond (H-bonding). Water at 25C in mesoporous
carbon (micro mesoporous) then water is of Type V, while Nitrogen at 77.4K is
Type IV. The nitrogen has large sections which are basically accessible via
necks. For water, the isotherm creeps along the P/Po. Basically Nitrogen forms a
wetting isotherm, and water on carbon is a non-wetting isotherm for a while.
Water has \textbf{delayed} micropore filling. For nitrogen, micropore pressure takes
place at very low pressures. Water is smaller than nitrogen, so it does enter
pores which nitrogen will always ignore. Since it is non wetting on carbon, the
micropore will fill \textbf{after} the mesopore. The hysteresis has been suspected to
be through a cooperative mechanism, that is through the hydrogen bonds. The pore
space fills with a dense phase of a vapor without any of it getting on the
surface. Essentially these are confined systems.

\subsection{Nitrogen}
\label{sec:orgfe6b6e7}
There is no characteristic microporous filling for nitrogen at 77K, this is
because the pore sizes are approximate. There are crosses, even when considering
DFT systems.
\subsection{Water}
\label{sec:org450b86b}
So in the same microporous systems, we can see that the non wetting system
behave differently. Essentially for these systems, the water isotherm is a
better classifier. The surface chemistry can be investigated by the adsorption
isoherms.

In the mercury symmetry operations, the meniscus is actually already present.
These other processes are dominated by the confinement effects. Therefore it is
not the standard wetting-dewetting situations.
Consider Periodic Mesoporous Organosilicas

Two PMO materials with identical pore size, different organic molecules lead to
different surface chemistry for the two PMOs, the differences in water
adsorption isoterms due to differences in surface chemistry.
Essentially, though we can get pore \textbf{volume}, we cannot get pore \textbf{size}.

Multiple runs of reversible and irreversible adsorption.
Dehydrogxilated silica is hydrophobic. Pure siilica is hydrophobic.
Essentially we can track the hydroxylation on the surface. So we start with an
irreversible isotherm hysteresis, which increasing becomes more likea
traditional revirsible hysteresis. This is also due to the thermal treatment of
the silica itself. Only above 500C will silica start to de-hydroxylate. This
also makes the pore loose structural integrity, since by then you'd be over the
softening point, if not the melting point. Typically, organic materials can be
used instead of water vapor. e.g. RH sensors are often mesopourous silica. In
silicas this process modifies the optical properties as well.

Email: [[\url{mailto:martin.thomas@anton-paar.com}][martin.thomas[at]anton-paar.com]]


J. Rathousky, M. Thommes, Stud. Surf. Sci. Catal. 2007
Morell, J;Wolter, G; Froeba M
Thommes M; Morell J; Cychosz, K A; Froeba, M
Zhang K. et al, languir 2012, 28, 8664-8673
Thommes, M; Morell, J.; Cychosz, K.A; Froeba, M. Languir 2013
Thommes, M, C. moriay, R. Ahmad, J.P. Joly, Adsorption 2011
\end{document}
